\documentclass{article}
\usepackage[utf8]{inputenc}

\title{Final Progress Report}
\author{Mike Fang\\ mf647 \and Andrei Shpilenok\\ ais76}

\begin{document}
\maketitle
\section*{Vision}
Our vision of unifying content and styling natively has not
changed with this sprint. Our goal is to develop a tech demo of a language
for the web which brings together CSS and HTML and absolves
javascript of the responsibility of having to do everything. We also aim to improve 
styling in the web by integrating constraints with our language.
project to consist of two main parts:
\begin{enumerate}
	\item Designing and Parsing the Language
	\item Creating a Dummy Browser to Display the Result of the Language
\end{enumerate}

\section*{Progress}

We spent this sprint finishing up implementing existing features, polishing, adding text rendering, and some nice-to-haves.

On the finishing up end, we finished implementing the constraint solver and style tree builder. This
had ended up being a lot more work than we had expected, so we are happy to have been able to finish
it this sprint. 

We also implemented text rendering, which was challenging in a similar way to first setting up the
window - there are very many ways to do it, each very different with their own upsides and downsides.
After trying multiple solutions, we settled on one that we were happy with.

We also added a couple miscellaneous items like support for color hex codes and trait aliases.

\section*{Activity}

\subsection*{Andrei}
\begin{enumerate}
	\item Implemented text rendering
	\item Expanded parser to add aliase and hex code support
    \item Wrote the progress report
\end{enumerate}

\subsection*{Mike}
\begin{enumerate}
	\item Finished support for full constriant styling capability
    \item Polished the render tree module
\end{enumerate}

\section*{Productivity}
We had yet again underestimated the challenges ahead of us. Adding text rendering and finishing the 
constraint styling alone was as much work as we had expected the more complete end of good scope to
be, so we're very happy with having done that but also implementing additional parser features.

\section*{Grade}

Given the difficutly of finishing the render tree and adding text rendering, we think our work would
be accurately reflected by an excellent scope grade. Although we had not implemented some of the 
features we had hoped to have done for excellent scope, such as built-in tags and functions, we think
the work required to add the features that we did was comparable. We chose to add these features
rather than the ones we outlined as goals in our previous progress report because we felt they were
more important for a working demo. 

\section*{Possible Next Steps}

There are more possible next steps than one can possibly enumerate! To no one's suprise, implementing
a browser is an insane amount of work, and there is always more to be done.

If we were interested in fleshing this out to be a complete language, there is a lot that could be
done, such as adding support for various data types specifying size and conversion between them,
built-in traits that specify styling and positioning functionality, as well as support for functions
and variables. 

One could also keep trying to innovate with the syntax and semantics of the language, as there are
always improvements to be had in the design of any system. 

Overall, though, we had achieved the progress required for a complete and fully-functional
tech demo of the features that are unique to our design and idea, and we are happy with it.
\end{document}
