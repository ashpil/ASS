\documentclass{article}
\usepackage[utf8]{inputenc}

\title{Beta: Progress Report}
\author{Mike Fang\\ mf647 \and Andrei Shpilenok\\ ais76}

\begin{document}
\maketitle
\section*{Vision}
Our vision of unifying content and styling natively has not
changed with this sprint. Our goal is to develop a fully fleshed out language
for the web which brings together CSS and HTML and absolves
javascript of the responsibility of having to do everything. We also aim to improve 
styling in the web by integrating constraints with our language. We envision the
project to consist of two main parts:
\begin{enumerate}
	\item Designing and Parsing the Language
	\item Creating a Dummy Browser to Display the Result of the Language
\end{enumerate}

After creating a dummy, end-to-end implementation, we realized we will not
be able realize a fully-fleshed out union of CSS and HTML, as we will
not have the time to implement all features. However,
we believe we will still be able to implement a language which will be
able to structure and describe content in a more concise way.

\section*{Progress}

This sprint we have completed an end-to-end, tracer-bullet implementation.
The user is now able to draw nodes using the
language. The nodes show up on the canvas as green rectangles.

The way this works is as follows:
\begin{enumerate}
    \item Parsed Source Code is transformed into a Stylesheet and DOM
    \item DOM and Stylesheet are combined into the Style Tree
          (Now we know what nodes we have to paint and their corresponding styles)
    \item Walk the Style Tree and each nodes constraint styles to create a Constraint Variable Pool which is a map from
          Style Tree Node IDs to attribute names to constraint variables
    \item Walk the Style Tree again and gather all constraint objects into the
          constraint solver
    \item Construct the Render Tree which consists of identical nodes to the Style Tree,
          but they now store the direct values of attributes which are retrieved from
          the solver
    \item The Render Tree is then passed to the Scene module to be rendered
\end{enumerate}

\section*{Activity}

\subsection*{Andrei}
\begin{enumerate}
	\item Wrote the Graphics Portion of the Blog Post
	\item Implemented the Display Module
\end{enumerate}

\subsection*{Mike}
\begin{enumerate}
	\item Wrote the Render Tree portion of the Blog Post
	\item Wrote the Progress Report
	\item Implemented the Render Tree Module
\end{enumerate}

\section*{Productivity}
We were very productive this sprint and accomplished our main goal
of creating an end-to-end implementation which can take our web DSL and
transform it into visible nodes in a dummy browser. However, we severely
overestimated the difficulty of creating the render tree and
being able to paint to canvas. As a result, we did not get to implementing
tags

\section*{Grade}

While we did not manage to reach our excellent-scope goal
of implementing tags, the difficulty of reaching an end-to-end dummy
implementation was more challenging than we expected and we believe we
still achieved Excellent scope when we factor in the amount of work
we did. Constructing the render tree was especially challenging
and we ran into many roadblocks, which often resulted in refactoring
previously written code.

\section*{Goals}

\subsection*{Satisfactory}
\begin{enumerate}
	\item Achieve full constraint styling capability with well-defined
            attributes as well as implementation of non-constraint-based
            styles
\end{enumerate}

\subsection*{Good}
\begin{enumerate}
	\item Be able to draw text on the canvas
	\item Users will be able to use pre-implemented tags in the code to structure content
\end{enumerate}

\subsection*{Excellent}
\begin{enumerate}
	\item Users will be able to create custom tags
	\item Users will be able to use functions and variables 
		in the styling portion of the language to structure and
		style data
\end{enumerate}
\end{document}
